\documentclass[a4paper, 14pt]{extarticle}
\usepackage[utf8]{inputenc}
\usepackage[russian]{babel}
\usepackage{graphicx}
\usepackage{listings}
\usepackage{color}
\usepackage{amsmath}
\usepackage{pgfplots}
\usepackage{url}
% подключаем hyperref (для ссылок внутри  pdf)
\usepackage[unicode, pdftex]{hyperref}
\usepackage[T2A]{fontenc}
\usepackage[utf8]{inputenc}
\lstset{tabsize=2,
    breaklines,
    columns=fullflexible,
    flexiblecolumns,
    numbers=left,
    keepspaces=true,
    numberstyle={\footnotesize},
    extendedchars=\true
}
\lstdefinelanguage{MyC}{
  language=C++,
  ndkeywordstyle=\color{darkgray}\bfseries,
  identifierstyle=\color{black},
  morecomment=[n]{/**}{*/},
  commentstyle=\color{blue}\ttfamily,
  stringstyle=\color{red}\ttfamily,
  morestring=[b]",
  showstringspaces=false,
  morecomment=[l][\color{gray}]{//},
  keepspaces=true,
  escapechar=\%,
  %texcl=⟨true|false⟩
}
%\usepackage[russian,russian,english]{babel}
\frenchspacing                    % ставим пробелы в соответствии с французским стилем
\DeclareGraphicsExtensions{.pdf,.png,.jpg,.svg}
\usepackage{titlesec}
\usepackage[russian]{babel}
\usepackage{algpseudocode}
\usepackage{caption}
\usepackage{setspace}
\usepackage[linesnumbered,boxed]{algorithm2e}
\DeclareCaptionFont{white}{\color{white}} %% это сделает текст заголовка белым
%% код ниже нарисует серую рамочку вокруг заголовка кода.

\linespread{1}

\DeclareCaptionFormat{listing}{\colorbox{gray}{\parbox{\textwidth}{#1#2#3}}}
\captionsetup[lstlisting]{format=listing,labelfont=white,textfont=white} 

\begin{document}
	\begin{titlepage}
		\begin{center}
			\begin{LARGE}
				Отчет по лабораторной работе №2\\
					по курсу "Анализ алгоритмов"\\
					по теме "Алгоритм умножения матриц \\
					Коперсента-Винограда"
			\end{LARGE}
		
			\begin{Large}
				\vspace{10cm}
				Студент: Зейналов З. Г. ИУ7-51\\
					Преподаватель: Волкова Л.Л.\\
				
				\vspace{5cm}2019 г.				   
			\end{Large}
			
		\end{center}
		 
	\end{titlepage}

\tableofcontents
	
\newpage
\section*{Введение}
\addcontentsline{toc}{section}{Введение}
\hspace{1cm} Целью данной лабораторной работы является исследование существующих алгоритмов умножение матриц и трудоемкости их вычислений.\\* Выберем следующую модель вычислений:\\
\begin{enumerate}
\item Трудоемкость базовых операций: \\ Операции +, -, *, /, $\div$, <, <=, >=, >, ==, !=, [ ], +=, -= - имеют стоимость 1 
\item Трудоемкость условия перехода  - 0, при этом расчет условия оцениванеим:\\
\lstset{language = C++}
\begin{lstlisting}
 if ( N % 2 == 1)
{
	// Тело1
}
else
{
	// Тело2
}
\end{lstlisting}
f__if  =  f_условия +
\left\{ 
\begin{eqnarray}
f_т1 , при нечетном N  \\
f_т2 , при четном
\end{eqnarray}}
\item 
\end{enumerate}





\end{document}